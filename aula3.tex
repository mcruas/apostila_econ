\section{Aula 3}


\subsection{Introdução ao problema de fluxo de potência}

Como os elementos vistos na aula anterior se combinam, formando um
sistema elétrico. Objetivo apresentar as equações de fluxo de potência 

Introduziremos a seguir os conceitos das leis de Kirchoff, que valem
tanto para circuitos pequenos como para sistemas elétricos inteiros.
A\textbf{ Primeira Lei de Kirchoff} enuncia que em qualquer nó, a
soma das correntes que deixam este nó (aquelas cujas apontam para
fora do nó) é igual a soma das correntes que chegam até ele. Observe
a figura \ref{fig:Exemplo-de-circuito}, a equação para o nó $a$
para satisfazer a primeira lei de Kirchoff é 
\[
i_{1}+i_{2}=i_{3}.
\]


\begin{figure}
\begin{centering}
\includegraphics{anexos/aula3_circuito}
\par\end{centering}

\protect\caption{\label{fig:Exemplo-de-circuito}Exemplo de circuito}
\end{figure}


Já a \textbf{Segunda Lei de Kirchoff} estabelece que a soma algébrica
das forças eletromotrizes (f.e.m) em qualquer malha é igual a soma
algébrica, considerando os sentidos, das quedas de tensão (ou dos
produtos $i\cdot R$) contidos na malha. Observando a malha da esquerda
do circuito apresentado na figura \ref{fig:Exemplo-de-circuito},
essa lei equivale a equação
\[
E_{1}+R_{1}\cdot i_{1}-R_{2}\cdot i_{2}-E_{2}+R_{1}\cdot i_{1}=0.
\]


Para modelar a rede, considera-se um modelo em que as fontes de tensão
foram transformadas em fontes de corrente em paralelo. %
\begin{comment}
VERIFICAR
\end{comment}


A figura \ref{fig:modelagem-sistema} mostra um diagrama da representação
de um sistema elétrico como um circuito, cujos elementos são impedâncias
(ou admitâncias, que é o inverso da impedância). Esse sistema pode
ser modelado com as seguintes equações:

\[
\begin{aligned}\dot{I}_{k}=\dot{I}_{km}+\dot{I}_{km}^{sh}, & (\mbox{Corrente injetada na barra \ensuremath{k})}\\
\dot{I}_{m}=\dot{I}_{mk}+\dot{I}_{mk}^{sh}, & (\mbox{Corrente injetada na barra \ensuremath{m})}\\
y_{km}=\frac{1}{R_{km}+jx_{km}}, & (\mbox{Admitância da linha)}\\
\dot{I}_{km}=(\dot{V}_{k}-\dot{V}_{m})\cdot y_{km}, & (\mbox{Corrente na linha)}\\
\dot{I}_{km}^{sh}=j\cdot\dot{V}_{k}\cdot b_{km}^{sh}, & (\mbox{Corrente no ramo shunt)}\\
\dot{I}_{km}=(\dot{V}_{k}-\dot{V}_{m})\cdot y_{km}, & (\mbox{Corrente na linha)}\\
\dot{I}_{km}^{sh}=j\cdot\dot{V}_{k}\cdot b_{km}^{sh}, & (\mbox{Corrente injetada na barra \ensuremath{m})}
\end{aligned}
\]


\begin{comment}
Ramo shunt. Que é? É tipo terra?
\end{comment}


\begin{figure}
\begin{centering}
\includegraphics{anexos/aula3_circuito2}
\par\end{centering}

\protect\caption{\label{fig:modelagem-sistema}Exemplo de modelagem de linha de transmissão}
\end{figure}


Matematicamente, as correntes injetadas nas barras podem ser obtidas
por:

\[
\left[\begin{array}{c}
i_{k}\\
i_{m}
\end{array}\right]=\left[\begin{array}{cc}
y_{km}+jb_{km}^{sh} & -y_{km}\\
-y_{km} & y_{km}+jb_{km}^{sh}
\end{array}\right]\left[\begin{array}{c}
\dot{V}_{k}\\
\dot{V}_{m}
\end{array}\right]\quad\longrightarrow\quad I=\left[Y_{bus}\right]\cdot U,
\]
em que $I$ representa o vetor de corrente nodal e $U$ o vetor de
tensão nodal. A regra da formação da matriz $Y_{bus}$ são:
\[
Y_{kk}=\sum_{m=1}^{n}\left(y_{km}+jb_{km}^{sh}\right)
\]


\[
Y_{km}=-y_{km}
\]
em que $Y$ é a matriz de admitância de barra e $n$ o número de barras
do sistema. Utiliza-se como notação a letra maiúscula com índice duplo
($Y_{kk},$ por exemplo) para representar um elemento da matriz $Y$,
enquanto a letra minúscula com índice duplo ($y_{km},$ e.g.) representa
a admitância do elemento do sistema. 

Assim, para um sistema de $n$ barras, matriz $Y_{bus}$ pode ser
escrita da seguinte forma:

\[
\left[Y_{bus}\right]=\left[\begin{array}{ccc}
Y_{11} & \cdots & Y_{1n}\\
\vdots & \ddots & \vdots\\
Y_{n1} & \cdots & Y_{nn}
\end{array}\right]=\left[\begin{array}{ccc}
G_{11}+jB_{11} & \cdots & G_{1n}+jB_{1n}\\
\vdots & \ddots & \vdots\\
G_{n1}+jB_{n1} & \cdots & G_{nn}+jB_{nn}
\end{array}\right]
\]
\[
\left[Y_{bus}\right]=\left[\begin{array}{ccc}
G_{11} & \cdots & G_{1n}\\
\vdots & \ddots & \vdots\\
G_{n1} & \cdots & G_{nn}
\end{array}\right]+j\left[\begin{array}{ccc}
B_{11} & \cdots & B_{1n}\\
\vdots & \ddots & \vdots\\
B_{n1} & \cdots & B_{nn}
\end{array}\right],
\]
em que $G$ é a matriz de condutância nodal e $B$ a matriz de suceptância
nodal.

A potência complexa injetada em uma barra é dada por 

\begin{equation}
S_{k}=\dot{V}_{k}\cdot\dot{I}_{k}^{*},\label{eq:aula3_sk}
\end{equation}
em que 
\begin{equation}
\dot{I}_{k}=\sum_{m=1}^{n}Y_{km}\cdot\dot{V}.\label{eq:aula3_iksum}
\end{equation}
Substituindo \ref{eq:aula3_iksum} em \ref{eq:aula3_sk}, obtemos
que 
\[
S_{k}=\dot{V}_{k}\cdot\left(\dot{I}_{k}=\sum_{m=1}^{n}Y_{km}\cdot\dot{V}\right)^{*}.
\]
Reescrevendo através de fasores obtemos que

\[
\begin{aligned}S_{k} & =V_{k}\angle\theta_{k}\cdot\left[\sum_{m=1}^{n}(G_{km}+jB_{km})\cdot V_{m}\angle\theta_{m}\right]^{*}\\
 & =V_{k}\angle\theta_{k}\cdot\left[\sum_{m=1}^{n}(G_{km}-jB_{km})\cdot V_{m}\angle-\theta_{m}\right]\\
 & =\left[\sum_{m=1}^{n}(G_{km}-jB_{km})\cdot V_{k}\cdot V_{m}\angle\underbrace{\theta_{k}-\theta_{m}}_{\mbox{chamando de \ensuremath{\theta_{km}}}}\right]\\
 & =\left[\sum_{m=1}^{n}\underbrace{G_{km}\cdot V_{k}\cdot V_{m}\angle\theta_{km}}_{G'}-\underbrace{jB_{km}\cdot V_{k}\cdot V_{m}\angle\theta_{km}}_{B'}\right]
\end{aligned}
\]
\begin{comment}
24:16
\end{comment}
A potência injetada em uma barra é dada também por 
\[
S_{k}=P_{k}+jQ_{k}.
\]
$P_{k},$ o componente real, chamado de parte da potência ativa, é
aquele que realiza trabalho no sistema, que acende lâmpadas, fornece
energia para motores, etc. Já $Q_{k}$, o componente imaginário, chamado
de parte reativa, é aquela que sustenta o sistema, mantendo os níveis
de tensão. As equações das potências injetadas em cada barra $k$
são dadas por
\[
\begin{aligned}P_{k} & =V_{k}\sum_{m=1}^{n}V_{m}\cdot(G_{km}\cdot\cos\,\theta_{km}+B_{km}\cdot\sin\,\theta_{km}),\qquad\forall k=1,\dots,n\\
Q_{k} & =V_{k}\sum_{m=1}^{n}V_{m}\cdot(G_{km}\cdot\cos\,\theta_{km}-B_{km}\cdot\sin\,\theta_{km}),\qquad\forall k=1,\dots,n
\end{aligned}
\]
Assim, para cada barra temos 2 equações, o que gera um total de $2n$
equações para todo o sistema, em que as variáveis podem ser $P_{k}$,
$Q_{k}$, $\theta_{k}$ e $V_{k}$. $G_{k}$ é um parâmetro que varia
de acordo com o tipo de linha tratado e $V_{k}$, que representa a
magnitude da potência.%
\begin{comment}
\end{comment}


As barras do sistema elétrico podem ter seus valores fixos ou estarem
soltos para que se ajustem às restrições impostas pelo fluxo de potência
necessário para que as equações sejam satisfeitas. 

\begin{figure}
\begin{centering}
\includegraphics{anexos/aula3_barras}
\par\end{centering}



\protect\caption{\label{fig:modelagem-sistema-1}Exemplo de modelagem de linha de transmissão}
\end{figure}




\subsection{Método Newton-Raphson}
O Método Newton-Raphson tem como objetivo estimar as raízes de uma
função, escolhendo uma aproximação inicial para esta e após a escolha
calcular a equação tangente da função neste ponto e a interseção dela
com o eixo das abcissas, a fim de encontrar uma melhor aproximação
para a raiz. O processo é repetido, criando assim um método iterativo
para encontrar a raiz da função. O algoritmo para o caso univariado é apresentado a seguir.


\begin{framed} %Box algoritmo
    \textbf{O algoritmo de Newton-Raphson}
    
    \begin{enumerate}
    \item Arbitrar uma condição inicial $(x_{i}-x_{0})$ e fixar $i=0$;
    
    \item Calcular $f(x_{i})$ e verificar a convergência. Se $|f(x_{i})|\leq \varepsilon$, então parar.
    
    \item Linearizar a função em torno de $(f(x_{i}),x_{i})$ e igualar a função
    a zero para estabelecer o passo $(\triangle x_{i}=x_{i+1}-x_{i})$
    e novo ponto $(x_{i+1})$.
    $$f(x)=f(x_{i})+f^{'}(x_{i})\cdot(x_{i+1}-x_{i})=0$$
    $$x_{i+1}=x_{i}-\frac{f(x_{i})}{f^{'}(x_{i})}$$
    
    \item Fazer $i=i+1$e voltar ao passo 2.
    
    \end{enumerate}
    
    Uma representação  gráfica do algoritmo é mostrada na figura \ref{fig:aula3_7}.
    
\end{framed}

\begin{figure}[H]
\begin{centering}
\includegraphics[scale=0.8]{aula3_7}\protect\caption{\label{fig:aula3_7} Algoritmo do Método de Newton  }
\end{centering}
\end{figure}

O método de Newton se baseia na expansão por \textbf{séries de Taylor} que, por definição, é uma forma de aproximar uma função $f$ ao redor de um ponto $x_{0}$ através da soma de polinômios.
Se uma função e suas derivadas até a ordem $n+1$ forem contínuas
em um intervalo contendo $x$ e $x_{0}$, então a expansão de Taylor é dada por: 

\[
f(x)=f(x_{0})+f^{'}(x_{0})\cdot(x-x_{0})+\frac{1}{2!}\cdot f^{''}(x_{0})\cdot(x-x_{0})^{2}+...+\frac{1}{n!}\cdot f^{(n)}(x_{0})\cdot(x-x_{0})^{n}+R_{n}.
\]

A representação gráfica dos termos da série de Taylor é dada pela Figura \ref{fig:aula3_2}.

\begin{figure}[H]
\begin{centering}
\includegraphics{aula3_2}\protect\caption{\label{fig:aula3_2} Série de Taylor }
\end{centering}
\end{figure}


Truncando a série no termo de 1ª ordem (ou seja, descartando todos os termos de ordem superior a 2), temos: 
\[
	f(x)=f(x_{0})+f^{'}(x_{0})\cdot(x-x_{0}).
\]
Tomando a expansão em torno de $x_0$, a interseção desta reta com o eixo das abcissas é encontrada em $x'$ resolvendo a equação: 
\[
	x' = x_0 - \dfrac{f(x_0)}{f'(x_0)} .
\]
Assim, o algoritmo funciona tomando o $x'$ como o valor de $x_{i+1}$ depois da $i$-ésima iteração, e continuando o algoritmo até atender um critério de parada. 
%  Como queremos (escolha inicial):
%\[
%x_{1}=x_{0}-\frac{f(x_{0})}{f^{'}(x_{0})},
%\]


\begin{exemplo}
Seja a função $f(x)=x^{2}-6\cdot\sin(x)$ com $x\in[1;3]$. Uma maneira de se encontrar a solução é plotar a função e procurar a solução visualmente, conforme na figura \ref{fig:aula3_1}. 
\begin{figure}[H]
\begin{centering}
\includegraphics{aula3_1}\protect\caption{\label{fig:aula3_1} Solução gráfica }
\end{centering}
\end{figure}

Considerando
uma escolha inicial de $x_{0}=2,5$, vamos utilizar o método de Newton-Raphson para aproximar uma raiz da função $f$ com uma tolerância $\varepsilon=0,05$.  Para a primeira iteração, devemos procurar
\[
x_{1}=x_{0}-\frac{f(x_{0})}{f^{'}(x_{0})},
\]
A função avaliada em $x_{0}=2,5$, com as derivadas em torno do ponto resultam
em:
\begin{eqnarray*}
f(x_{0}) & = & f(2,5)=2,6592 \\
f^{'}(x_{0}) & = & f^{'}(2,5)=9,8069
\end{eqnarray*}
Assim, substituindo os valores na equação para $x_{0}=2,5$, obtém-se uma aproximação da função $f$ em torno de $x_0$:
\[
f(x)\cong2,66+9,81\cdot(x-2,5).
\]

\begin{figure}[H]
\begin{centering}
\includegraphics[scale=0.8]{aula3_3}\protect\caption{\label{fig:aula3_3} Gráfico  $x_{0}=2,5$  }
\end{centering}
\end{figure}
Na figura \ref{fig:aula3_3}, a reta azul representa a função original do problema e a verde é a expansão em série de Taylor linearizada no polinômio de primeira ordem. O ponto $x_1$ é encontrado através da expressão
\[
x_{1}=2,5-\frac{2,6562}{9,8069}\cong2,23.
\]
Usando o ponto $x_{1}=2,23$ na função $f(x)=x^{2}-6\cdot\sin(x)$:

\[
| f(2,23)|=|2,23^{2}-6\cdot\sin(2,23)|=|0,23|>0,05
\]
A solução $x_{1}=2,23$ não satisfaz a restrição do problema. Utilizamos o $x_1$ para a próxima iteração (ver figura \ref{fig:aula3_4}), temos que 
\begin{figure}[H]
\begin{centering}
\includegraphics[scale=0.8]{aula3_4}\protect\caption{\label{fig:aula3_4} Gráfico com a tolerância de $x_{0}=2,5$  }
\end{centering}
\end{figure}
O ponto $x_{1}=2,23$ é o ponto de partida para a segunda iteração.
Linearizando em torno de $x_{1}=2,23$ (Gráfico \ref{fig:aula3_5}):
\begin{figure}[H]
\begin{centering}
\includegraphics[scale=0.8]{aula3_5}\protect\caption{\label{fig:aula3_5} Linearização no ponto $x_{1}=2,23$  }
\end{centering}
\end{figure}
$$f(x_{1})=f(2,23)=0,23,$$
$$f^{'}(x_{1})=f^{'}(2,23)=8,1319,$$
$$x_{2}=x_{1}-\frac{f(x_{1})}{f^{'}(x_{1})}.$$
Assim, temos que:
$$f(x)\cong0,23+8,1349\cdot(x_{2}-2,23)$$
$$x_{2}=2,23 - \frac{0,23}{8,81349} \cong 2,20.$$
Usando o ponto $(x_{2}=2,2)$ na função $f(x_{2}^{2})=x_{2}^{2}-6\cdot\sin x_{2}$:
\[
	|f(2,2)|=|2,2^{2}-6\cdot\sin(2,2)|=|0,03|<0,05.
\]
Portanto, $x_{2}=2,2$ satisfaz a solução do problema. Caso seja necessário
encontrar uma tolerância menor que $0,03$ o processo deve ser repetido.
Graficamente, para $x_{1}=2,23$:
\begin{figure}[H]
\begin{centering}
\includegraphics[scale=0.8]{aula3_6}\protect\caption{\label{fig:aula3_6} Gráfico para $x_{1}=2,23$  }
\end{centering}
\end{figure}







\begin{figure}[H]
\begin{centering}
\includegraphics[scale=0.8]{aula3_8}\protect\caption{\label{fig:aula3_8} Escolha inicial  }
\end{centering}
\end{figure}

\end{exemplo}


Ao aplicar o método de Newton, deve-se tomar cuidado. O método é eficiente quando o ponto inicial $x_0$ é relativamente próximo da solução, mas pode ser ruim quando o chute inicial não for bom. Observe a figura \ref{fig:aula3_8}. A derivada avaliada em $x_0$ é 0, assim $x_{1}=x_{0}-\frac{f(x_{0})}{f^{'}(x_{0})}$; como o termo $\frac{f(x_{0})}{f^{'}(x_{0})}$ vai para infinito, a solução nunca seria encontrada. 

O método de Newton é adequado para o problema do fluxo de potência. Como as soluções do problema ($V$ e $\theta$) normalmente são próximas de 1 (um) pu e 0 (zero) radianos, então a escolha da solução inicial será normalmente adequada.

A diferença do fluxo de potencial do exemplo anterior é que temos
um conjunto de funções para resolver.

Já o \textbf{método de Newton aplicado ao caso multivariável} deve considerar:
\[
\begin{matrix} 
F(x)= & [f_{ 1 } & { f }_{ 2 } & ... & { f }_{ n }{ ] }^{ T } 
\end{matrix}
\begin{matrix} 
x= & [x_{ 1 } & { x }_{ 2 } & ... & { x }_{ n }{ ] }^{ T } 
\end{matrix}
\]
Sendo que, $F(x)$ representa um vetor de funções e $x$ é um conjunto
de variáveis.


\begin{framed} %Box algoritmo multivariado
    \textbf{O algoritmo de Newton-Raphson multivariado}
    \begin{enumerate}
    \item Arbitrar uma condição inicial $(x^{(i)}=x^{(0)})$ e fixar $i=0$;
    
    \item Calcular $F(x^{(i)})$ e verificar a convergência. Se $\max| F(x^{(i)})|\leq \varepsilon,$ então parar.
    
    \item Linearizar a função em torno de $(F(x^{(i)}),x^{(i)})$ e igualar
    a função a zero para estabelecer o passo $(\triangle x^{(i)}=x^{(i+1)}-x_{i}^{(i)})$
    e novo ponto $(x^{(i+1)})$.
    Para o caso de apenas uma variável, temos que:
    \[
    f(x)=f(x_{i})+f^{'}(x_{i})\cdot(x_{i+1}-x_{i})=0
    \]
    \[
    x_{i+1}=x_{i}-\frac{f(x_{i})}{f^{'}(x_{i})}.
    \]
    Já para o caso multivariado, temos que:
    \[
    F(x)=F(x^{(i)})+[J(x^{(i)})]\cdot(x^{(i+1)}-x^{(i)})=0
    \]
    \[
    x^{(i+1)}=x^{(i)}-[J(x^{(i)})]^{-1}\cdot F(x^{(i)}),
    \]
    sendo que $J(x^{(i)})=[\frac{\partial F(x)}{\partial x}]$ é a matriz jacobiana
    de derivadas de $F(x)$ com relação à $x$.
    \item Fazer $i=i+1$ e voltar ao passo 2.
    \end{enumerate}
    
    
\end{framed}


Para calcular a matriz jacobiana, primeiro cria-se um vetor $F$ com as $n$ funções em questão:
\[
\begin{matrix} 
F(x)= & [f_{ 1 } & { f }_{ 2 } & \dots & { f }_{ n }{ ] }^{ T }, 
\end{matrix}
\]
em que $\begin{matrix} 
x= & [x_{ 1 } & { x }_{ 2 } & \dots & { x }_{ n }{ ] }^{ T } 
\end{matrix}$. A matriz jacobiana é dada pelas derivadas cruzadas da função $F$:
\[
	\left[J(x^{(i)})\right]=\left[\frac{\partial F(x)}{\partial x}\right]=\left[\begin{array}{ccc}
	\frac{\partial f_{1}}{\partial x_{1}} & \cdots & \frac{\partial f_{1}}{\partial x_{m}}\\
	\vdots & \ddots & \vdots\\
	\frac{\partial f_{n}}{\partial x_{1}} & \vdots & \frac{\partial f_{n}}{\partial x_{m}}
	\end{array}\right]
\]
As equações básicas do subproblema 1 (barras PQ+PV) a serem solucionadas
são:
\[
P_{k}=V_{k}\sum_{m=1}^{n}V_{m}(G_{km}\cos\theta_{km}+B_{km}\sin\theta_{km}),\forall k\in\{\Omega_{PQ},\Omega_{PV}\},
\]
\[
Qk=V_{k}\sum_{m=1}^{n}V_{m}(G_{km}\sin\theta_{km}+B_{km}\cos\theta_{km}),\forall k\in\{\Omega_{PQ}\},
\]
onde, $\Omega_{PQ}$ são os conjuntos de barras do tipo PQ, $\Omega_{PV}$ são os conjuntos de barras do tipo PV. Considerando que para algumas barras os valores de P e Q são conhecidos, os ``resíduos'' de potência são dados por:
\[
\Delta P_{k}=P_{k}^{(especificado)}-P_{k}^{(calculado)}(V,\theta),\forall k\in\{\Omega_{PQ},\Omega_{PV}\},
\]
\[
\Delta Q_{k}=Q_{k}^{(especificado)}-Q_{k}^{(calculado)}(V,\theta),\forall k\in\{\Omega_{PQ}\},
\]
em que $P_{k}^{(especificado)}$ são os valores de potência ativa (conhecidos) na
barra $k$ e $Q_{k}^{(especificado)}$ são os valores de potência ativa(conhecido) na barra $k$.

Neste caso: \todo{como apresentar isso aqui?}
\[
	\begin{matrix}
	F(\Delta { P },\Delta Q)= & [\Delta { P }_{ 1 } & \Delta { P }_{ 2 } & ... & \Delta { P }_{ n }^{ \quad } & \Delta { Q }_{ 1 } & \Delta { Q }_{ 2 } & \dots & { \Delta { Q } }_{ n }] 
	\end{matrix}
\]
\[
	\begin{matrix} 
	x= & [\theta _{ 1 } & \theta _{ 2 } & ... & \theta _{ n }^{ \quad } & V_{ 1 } & V_{ 2 } & \dots & { V }_{ n }] 
	\end{matrix}
\]

A matriz Jacobiana associada ao problema proposto é apresentada na Figura \ref{fig:aula3_9}.


\begin{figure}[H]
\begin{centering}
\includegraphics{aula3_9}\protect\caption{\label{fig:aula3_9} Matriz jacobiana  }
\end{centering}
\end{figure}

A seguir, calculamos os elementos da matriz jacobiana:
\[
\frac{\partial\triangle P_{k}}{\partial\theta_{m}}=\frac{\partial(P_{k}^{(especificado)}-P_{k}^{(calculado)})}{\partial\theta_{m}}=-\frac{\partial P_{k}^{(calculado)}}{\partial\theta_{m}},\qquad\forall m,k
\]
\[
\frac{\partial\triangle P_{k}}{\partial V_{m}}=\frac{\partial(P_{k}^{(especificado)}-P_{k}^{(calculado)})}{\partial V_{m}}=-\frac{\partial P_{k}^{(calculado)}}{\partial V_{m}},\qquad\forall m,k
\]
\[
\frac{\partial\triangle Q_{k}}{\partial\theta_{m}}=\frac{\partial(Q_{k}^{(especificado)}-Q_{k}^{(calculado)})}{\partial\theta_{m}}=-\frac{\partial Q_{k}^{(calculado)}}{\partial\theta_{m}},\qquad\forall m,k
\]
\[
\frac{\partial\triangle Q_{k}}{\partial V_{m}}=\frac{\partial(Q_{k}^{(especificado)}-Q_{k}^{(calculado)})}{\partial V_{m}}=-\frac{\partial Q_{k}^{(calculado)}}{\partial V_{m}},\qquad\forall m,k
\]


Como $P_{k}^{(especificado)}$ e
$Q_{k}^{(especificado)}$ são constantes a matriz Jacobiana pode ser reescrita como apresentamos a seguir:
\[
[J]=- \begin{bmatrix} \frac { \partial { P }^{ (calculado) } }{ \partial \theta } & \frac { \partial { P }^{ (calculado) } }{ \partial V } \\ \frac { \partial { Q }^{ (calculado) } }{ \partial \theta } & \frac { \partial { Q }^{ (calculado) } }{ \partial V } \end{bmatrix}_{ 2n \times 2n }=-{ \begin{bmatrix} H & N \\ M & L \end{bmatrix} }_{ 2n \times 2n }
\]
\[
H_{km}=\frac{\partial P_{k}^{(calculado)}}{\partial\theta_{m}}=V_{k}\cdot V_{m}\cdot\{G_{km}\cdot\sin(\theta_{km})-B_{km}\cdot\cos(\theta_{km})\},\forall k\neq m
\]


\[
H_{km}=\frac{\partial P_{k}^{(calculado)}}{\partial\theta_{k}}=-V_{k}^{2}\cdot B_{kk}-V_{k}\cdot\left[\sum_{m\in\Omega_{k}}V_{m}\cdot\{G_{km}\cdot\sin(\theta_{km})-B_{km}\cdot\cos(\theta_{km})\}\right]
\]


\[
N_{km}=\frac{\partial P_{k}^{(calculado)}}{\partial V_{m}}=V_{k}\cdot\{G_{km}\cdot\cos(\theta_{km})-B_{km}\cdot\sin(\theta_{km})\},\forall k,m
\]


\[
N_{kk}=\frac{\partial P_{k}^{(calculado)}}{\partial V_{k}}=-V_{k}\cdot G_{kk}+\left[\sum_{m\in\Omega_{k}}V_{m}\cdot\{G_{km}\cdot\cos(\theta_{km})-B_{km}\cdot\sin(\theta_{km})\}\right]
\]


\[
M_{km}=\frac{\partial Q_{k}^{(calculado)}}{\partial\theta_{m}}=V_{k}\cdot V_{m}\cdot\{G_{km}\cdot\sin(\theta_{km})-B_{km}\cdot\cos(\theta_{km})\},\forall k\neq m
\]


\[
M_{kk}=\frac{\partial Q_{k}^{(calculado)}}{\partial\theta_{m}}=-V_{k}^{2}\cdot G_{kk}\cdot\left[\sum_{m\in\Omega_{k}}V_{m}\cdot\{G_{km}\cdot\cos(\theta_{km})+B_{km}\cdot\sin(\theta_{km})\}\right]
\]


\[
L_{km}=\frac{\partial Q_{k}^{(calculado)}}{\partial V_{m}}=V_{k}\cdot\{G_{km}\cdot\sin(\theta_{km})-B_{km}\cdot\cos(\theta_{km})\},\forall k\neq m
\]


\[
L_{kk}=\frac{\partial Q_{k}^{(calculado)}}{\partial V_{k}}=-V_{k}\cdot B_{kk}+\left[\sum_{m\in\Omega_{k}}V_{m}\cdot\{G_{km}\cdot\sin(\theta_{km})-B_{km}\cdot\cos(\theta_{km})\}\right],\forall k,m
\]
em que $\Omega_{k}$ são as barras vizinhas à barra k, incluindo a barra k.

\subsection{Aplicando o método de Newton para o problema do fluxo de potência}

Aplicando no exercício proposto anteriormente, calcular o resultado
do fluxo de potência.
\begin{figure}[H]
\begin{centering}
\includegraphics{aula3_10}\protect\caption{\label{fig:aula3_10} Exemplo }
\end{centering}
\end{figure}

\[
P_{1}^{calc}=1,1[1,1(0,33)+V_{2}(-0,33\cos\theta_{2}-3,3\sin\theta_{2})]
\]


\[
Q_{1}^{calc}=1,1[1,1(0,33)+V_{2}(-0,33\sin\theta_{2}-3,3\cos\theta_{2})]
\]


\[
P_{2}^{calc}=V_{2}\cdot[1,1\cdot(-0,33\cdot\cos\theta_{2}+B_{21}\cdot\sin\theta_{2})+V_{2}\cdot(1,706)+1,05\cdot(1,376\cdot\cos\theta_{32}+4,587\cdot\sin\theta_{32})]
\]


\[
Q_{2}^{calc}=V_{2}\cdot[1,1\cdot(1,706\cdot\sin\theta_{2}+7,887\cdot\cos\theta_{2})-V_{2}\cdot(7,887)+1,05\cdot(1,376\cdot\cos\theta_{32}+4,587\cdot\sin\theta_{32})]
\]


\[
P_{3}^{calc}=1,05\cdot[V_{2}\cdot(-1,376\cdot\cos\theta_{32}+4,587\cdot\sin\theta_{32})+1,05\cdot(1,376)]
\]


\[
Q_{3}^{calc}=1,05\cdot[V_{2}\cdot(-1,376\cdot\sin\theta_{32}+4,587\cdot\cos\theta_{32})-1,05\cdot(4,587)]
\]
Considerando:
\[
F(x)={ \begin{matrix} [\triangle { P }_{ 1 } & \triangle { P }_{ 3 } & \triangle { P }_{ 3 } & \triangle { Q }_{ 1 } & \triangle { Q }_{ 2 } & \triangle { Q }_{ 3 }] \end{matrix} }_{ 6 }^{ T }
\]
\[
x={ \begin{matrix} [\theta _{ 1 } & \theta _{ 2 } & \theta _{ 3 } & V_{ 1 } & V_{ 2 } & V_{ 3 }] \end{matrix} }_{ 6 }^{ T }
\]
\[
F(x)={ \begin{matrix} [\triangle { P }_{ 3 } & \triangle { P }_{ 3 } & \triangle { Q }_{ 2 } \end{matrix}] }_{ 3 }^{ T }
\]
\[
x={ \begin{matrix} [\theta _{ 2 } & \theta _{ 3 } & V_{ 2 }] \end{matrix} }_{ 3 }^{ T }
\]
\begin{enumerate}
\item Arbitrar uma condição inicial $(x^{(i)}=x^{(0)})$ e fixar $i=0$;
\[
{ x }^{ (0) }={ \begin{matrix} [0 & 0 & 1 \end{matrix}] }_{ 3 }^{ T }
\]
\item Calcular $F(x^{(i)})$ e verificar a convergência. Se $\max| F(x^{(i)})|\leq \varepsilon$,
parar.
\item Linearizar a função em torno de $(F(x^{(i)}),x^{(i)})$ e igualar
a função a zero para estabelecer o passo $(\varDelta x^{(i)}=x^{(i+1)}-x^{(i)})$
e o novo ponto $(x^{(i+1)})$.
\[
F(x)=F(x^{(i)})+[J(x^{(i)}]\cdot(x^{(i+1)}-x^{(i)})=0
\]
 
\[
x^{(i+1)}=x^{(i)}+[J(x^{(i)}]^{-1}\cdot F(x^{(i)})
\]


Para solucionar o problema proposto devemos determinar a matriz jacobiana:
\begin{figure}[H]
\begin{centering}
\includegraphics{aula3_11}\protect\caption{\label{fig:aula3_11} Matriz jacobiana do exemplo }
\end{centering}
\end{figure}
O problema apresentado faz com que, para o subproblema 1:
\[
F(x)={ \begin{matrix} [\triangle { P }_{ 3 } & \triangle { P }_{ 3 } & \triangle { Q }_{ 2 } \end{matrix}] }_{ 3 }^{ T }
\]
\[
x={ \begin{matrix} [\theta _{ 2 } & \theta _{ 3 } & V_{ 2 }] \end{matrix} }_{ 3 }^{ T }
\]
Barra 1: $V\theta$(slack bus)

Barra 2: $PQ$(barra de carga)

Barra 3: $PV$(barra de controle de tensão)
\item Fazer $i=i+1$ e voltar ao passo 2.

\end{enumerate}

A matriz Jacobiana par ao problema proposto fica:

\begin{figure}[H]
\begin{centering}
\includegraphics{aula3_12}\protect\caption{\label{fig:aula3_12} Matriz jacobiana }
\end{centering}
\end{figure}
Sendo

$n_{pq}$-Número de barras $PQ$;

$n_{pv}$-Número de barras $PV$;
De outra forma:
\begin{figure}[H]
\begin{centering}
\includegraphics{aula3_13}\protect\caption{\label{fig:aula3_13} Matriz jacobiana }
\end{centering}
\end{figure}

A solução do problema proposto depois de quatro iterações:

\begin{tabular}{|c|c|c|c|c|c|}
\hline 
Barras & V (p.u) & $P_{G}$ & $P_{D}$ & $Q_{G}$ & $Q_{D}$\tabularnewline
\hline 
\hline 
1 (Ref.) ($\theta_{1}=0)$ & 1,1 & - & 0 & - & 0\tabularnewline
\hline 
2 (PQ) & - & 0 & 1 & 0 & 0,4\tabularnewline
\hline 
3 (PQV) & 1,05 & 0,6 & 0,2 & - & 0,05\tabularnewline
\hline 
\end{tabular}
\begin{figure}[H]
\begin{centering}
\includegraphics{aula3_14}\protect\caption{\label{fig:aula3_14} Exemplo }
\end{centering}
\end{figure}

$z_{12}=-0,03+0,3j$

$z_{23}=-0,06+0,2j$
\[
{ x }^{ (4) }={ \begin{matrix} { [\theta _{ 2 } }^{ (4) } & { \theta _{ 3 } }^{ (4) } & { V_{ 2 } }^{ (4) }] \end{matrix} }_{ \quad }^{ T }={ \begin{matrix} [-0,1616\quad rad & -0,0958\quad rad & 0,993] \end{matrix} }_{ \quad }^{ T }\quad 
\]
\[
{ x }^{ (4) }={ \begin{matrix} { [\theta _{ 2 } }^{ (4) } & { \theta _{ 3 } }^{ (4) } & { V_{ 2 } }^{ (4) }] \end{matrix} }_{ \quad }^{ T }={ \begin{matrix} [-{ 9,263 }^{ \circ } & { -0,0958 }^{ \circ } & 0,993] \end{matrix} }_{ \quad }^{ T }\quad 
\]

$\varepsilon=0,000001$







